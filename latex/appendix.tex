\section{Appendice}
Questo appendice contiene l'implementazione dei coponenti notevoli utilizzati negli esercizi precedenti.

\subsection*{Multiplexer 4:1}
\begin{code}
    \inputminted{vhdl}{vhdl/mux4_1.vhd}
    \caption{Implementazione del multiplexer 4:1}
    \label{cod:mux_4_1}
\end{code}

\subsection*{Demultiplexer 1:4}
\begin{code}
    \inputminted{vhdl}{vhdl/dmux1_4.vhd}
    \caption{Implementazione del demultiplexer 1:4}
    \label{cod:dmux_1_4}
\end{code}

\subsection*{Multiplexer 2:1}
\begin{code}
    \inputminted{vhdl}{vhdl/mux_2_1.vhd}
    \caption{Implementazione del multiplexer 2:1}
    \label{cod:mux_2_1}
\end{code}

\subsection*{Button Debouncer}
\begin{code}
    \inputminted{vhdl}{vhdl/Button_Debouncer.vhd}
    \caption{Implementazione del debouncer per i pulsanti}
    \label{cod:button_debouncer}
\end{code}

\subsection*{Flip-Flop D}
\begin{code}
    \inputminted{vhdl}{vhdl/flipflop_D.vhd}
    \caption{Implementazione del flip--flop D}
    \label{cod:flipflop_D}
\end{code}

\subsection*{Clock Divider}
\begin{code}
    \inputminted{vhdl}{vhdl/clock_divider.vhd}
    \caption{Implementazione del divisore di clock}
    \label{cod:clock_divider}
\end{code}

\subsection*{Contatore (fronte di discesa)}
\begin{code}
    \inputminted{vhdl}{vhdl/counter_fallingedge.vhd}
    \caption{Implementazione del contatore con Q sul fronte di discesa}
    \label{cod:counter_fallingedge}
\end{code}

\subsection*{Gestore degli anodi}
\begin{code}
    \inputminted{vhdl}{vhdl/anodes_manager.vhd}
    \caption{Implementazione del gestore degli anodi}
    \label{cod:anodes_manager}
\end{code}

\subsection*{Contatore (fronte di salita)}
\begin{code}
    \inputminted{vhdl}{vhdl/counter_risingedge.vhd}
    \caption{Implementazione del contatore con Q sul fronte di salita}
    \label{cod:counter_risingedge}
\end{code}

\subsection*{ROM di N locazioni da 8 bit}
\begin{code}
    \inputminted{vhdl}{vhdl/ROM_N.vhd}
    \caption{Implementazione della memoria ROM di N locazioni da 8 bit}
    \label{cod:ROM_N}
\end{code}

\subsection*{Memoria di N locazioni da 4 bit}
\begin{code}
    \inputminted{vhdl}{vhdl/MEM_N.vhd}
    \caption{Implementazione della memoria MEM di N locazioni da 4 bit}
    \label{cod:MEM_N}
\end{code}

\subsection*{Registro di 8 bit}
\begin{code}
    \inputminted{vhdl}{vhdl/register_8.vhd}
    \caption{Implementazione del registro di 8 bit}
    \label{cod:register_8}
\end{code}

\subsection*{Registro a scorrimento}
\begin{code}
    \inputminted{vhdl}{vhdl/shiftregister.vhd}
    \caption{Implementazione del registro a scorrimento di 16 bit}
    \label{cod:shiftregister}
\end{code}

\subsection*{Full Adder}
\begin{code}
    \inputminted{vhdl}{vhdl/full_adder.vhd}
    \caption{Implementazione del full adder}
    \label{cod:full_adder}
\end{code}

\subsection*{Ripple Carry Adder}
\begin{code}
    \inputminted{vhdl}{vhdl/ripple_carry_adder_8.vhd}
    \caption{Implementazione del ripple carry adder a 8 bit}
    \label{cod:ripple_carry_adder_8}
\end{code}

\subsection*{Adder-Subtractor}
\begin{code}
    \inputminted{vhdl}{vhdl/adder_subtractor.vhd}
    \caption{Implementazione dell'unità addizionatore/sottrattore a 8 bit}
    \label{cod:adder_subtractor}
\end{code}

\subsection*{Registro di M bit}
\begin{code}
    \inputminted{vhdl}{vhdl/register_M.vhd}
    \caption{Implementazione del registro di M bit}
    \label{cod:register_M}
\end{code}

\subsection*{Sommatore ad M bit}
\begin{code}
    \inputminted{vhdl}{vhdl/adder.vhd}
    \caption{Implementazione del sommatore a M bit}
    \label{cod:adder}
\end{code}

\subsection*{ROM di N locazioni da M bit}
\begin{code}
    \inputminted{vhdl}{vhdl/ROM_N_M.vhd}
    \caption{Implementazione della memoria ROM di N locazioni da M bit}
    \label{cod:ROM_N_M}
\end{code}

\subsection*{Memoria di N locazioni da M bit}
\begin{code}
    \inputminted{vhdl}{vhdl/MEM_N_M.vhd}
    \caption{Implementazione della memoria MEM di N locazioni da M bit}
    \label{cod:MEM_N_M}
\end{code}

\subsection*{ROM di 8 locazioni da 8 bit}
\begin{code}
    \inputminted{vhdl}{vhdl/ROM_8_8.vhd}
    \caption{Implementazione della memoria ROM di 8 locazioni da 8 bit}
    \label{cod:ROM_8_8}
\end{code}

\subsection*{Memoria di 8 locazioni da 8 bit}
\begin{code}
    \inputminted{vhdl}{vhdl/MEM_8_8.vhd}
    \caption{Implementazione della memoria MEM di 8 locazioni da 8 bit}
    \label{cod:MEM_8_8}
\end{code}

\subsection*{RS232RefComp}
\begin{code}
    \inputminted{vhdl}{vhdl/RS232RefComp.vhd}
    \caption{Implementazione dell'unità UART RS232}
    \label{cod:RS232RefComp}
\end{code}

\subsection*{Multiplexer 2:1 a 2 bit}
\begin{code}
    \inputminted{vhdl}{vhdl/mux_2_1_2bit.vhd}
    \caption{Implementazione del multiplexer 2:1 a 2 bit}
    \label{cod:mux_2_1_2bit}
\end{code}

\subsection*{Demultiplexer 1:2 a 2 bit}
\begin{code}
    \inputminted{vhdl}{vhdl/dmux_1_2_2bit.vhd}
    \caption{Implementazione del demultiplexer 1:2 a 2 bit}
    \label{cod:dmux_1_2_2bit}
\end{code}

\subsection{ROM di 8 locazioni da 4 bit}
\begin{code}
    \inputminted{vhdl}{vhdl/ROM_8_4.vhd}
    \caption{Implementazione della memoria ROM di 8 locazioni da 4 bit}
    \label{cod:ROM_8_4}
\end{code}

\subsection{Carry Lookahead Adder a 4 bit}
\begin{code}
    \inputminted{vhdl}{vhdl/carry_look_ahead_adder_4.vhd}
    \caption{Implementazione dell'adder a carry lookahead a 4 bit}
    \label{cod:carry_look_ahead_adder_4}
\end{code}
